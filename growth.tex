\documentclass{IEEEcsmag}

\usepackage[colorlinks,urlcolor=blue,linkcolor=blue,citecolor=blue]{hyperref}
\expandafter\def\expandafter\UrlBreaks\expandafter{\UrlBreaks\do\/\do\*\do\-\do\~\do\'\do\''\do\-}
\usepackage{upmath,color}

%%\usepackage{draftwatermark}
%%\SetWatermarkText{DRAFT}


\jvol{XX}
\jnum{XX}
\paper{8}
\jmonth{Month}
\jname{\textit{IEEE Annals of the History of Computing}}
\jtitle{\textit{IEEE Annals of the History of Computing}}
\pubyear{2025}

\newtheorem{theorem}{Theorem}
\newtheorem{lemma}{Lemma}


\setcounter{secnumdepth}{0}

\begin{document}

\sptitle{ARTICLE}

\title{The First Fifteen Years of Computing in New Zealand}

\author{Brian E. Carpenter}
\affil{University of Auckland, New Zealand}

\author{Sathiamoorthy Manoharan}
\affil{University of Auckland, New Zealand}

\author{Janet Toland}
\affil{Victoria University of Wellington, New Zealand}


\markboth{ARTICLE}{ARTICLE}

\begin{abstract}This article outlines the first fifteen years of modern computing in New Zealand ... 
\end{abstract}

\maketitle


\chapteri{M}odern computers arrived in New Zealand in 1960, little over a century after the first discernable information technology \cite{pioneers}. It then being a relatively isolated and small (2.5 million people) economy, it is feasible to track the dissemination and socio-economic influence of computing for the following period; this study runs from 1960 to about 1975. Two events in the mid-1970s signaled the end of an economic and social era that began after World War II, and can act as the end of our study: Britain joined the European Common Market in 1973, and New Zealand switched on a national Police Computer in 1976.

Several themes are used to organize this article:

\begin{itemize}
\item growth in numbers of computers and vendors
\item growth in employees (and male dominance?) (and Pākehā/Māori?)
\item types of users, including computing service bureaus
\item key areas where the technology was used
\item local contributions to the technology 
\item start of computing services and teaching in universities , technical colleges (and maybe schools?)
\item professionalization (NZCS)
\item commercial impact
\item social impact
\end{itemize}

\vspace*{-8pt}
\section{GROWTH IN NUMBERS AND EMPLOYMENT}


The delivery of one of the successors of the first computer in New Zealand. It is an ICL 1902A being delivered to the premises of Motor Specialties on Anzac Avenue, Auckland in 1969. Almost certainly, this was installed in the same room as its predecessors, an ICT 1201 and an ICT 1301 (Figure~\ref{MS1902A}).

\begin{figure}
\centerline{\includegraphics[width=18.5pc]{MotorSpec1902A-1969.jpg}}
\caption{\label{MS1902A}ICL 1902A Delivery in 1969. Courtesy Fletcher Archives, NZ.}\vspace*{-5pt}
\end{figure}

TBD
\vspace*{-8pt}
\section{CONCLUSION}

TBD

\vspace*{-8pt}
\section{ACKNOWLEDGMENTS}

TBD

%% style ieeetr doesn't display all details or alphabetise
%% style IEEEtran doesn't alphabetise
%% style plain doesn't meet IEEE requirements
\bibliographystyle{IEEEtranS}
\bibliography{growth}

\begin{IEEEbiography}{Brian E. Carpenter} is an Honorary Professor in the School of Computer Science, University of Auckland, New Zealand. He is interested in Internet protocol design and in computing history. He received a Ph.D. from the University of Manchester, U.K. and is a past Chair of the Internet Engineering Task Force. Contact: brian@cs.auckland.ac.nz\vspace*{8pt}
\end{IEEEbiography}

\begin{IEEEbiography}{Sathiamoorthy Manoharan} is a Senior Lecturer in the School of Computer Science, University of Auckland, New Zealand.
He is a senior member of IEEE. He is interested in computer systems, particularly with respect to performance and security. Contact: mano.manoharan@auckland.ac.nz.\vspace*{8pt}
\end{IEEEbiography}

\begin{IEEEbiography}{Janet Toland} is currently an Associate Professor in the
School of Information Management, Te Herenga Waka, Victoria
University of Wellington, Wellington, New Zealand. Her
research work is in the history of information systems. She
has a particular interest in social and ethical issues. She is
currently the Historian for the Association of Information
Systems. Contact her at Janet.Toland@vuw.ac.nz.\vspace*{8pt}
\end{IEEEbiography}



\end{document}

