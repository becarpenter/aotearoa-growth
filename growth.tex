\documentclass{IEEEcsmag}

\usepackage[colorlinks,urlcolor=blue,linkcolor=blue,citecolor=blue]{hyperref}
\expandafter\def\expandafter\UrlBreaks\expandafter{\UrlBreaks\do\/\do\*\do\-\do\~\do\'\do\''\do\-}
\usepackage{upmath,color}

%%\usepackage{draftwatermark}
%%\SetWatermarkText{DRAFT}


\jvol{XX}
\jnum{XX}
\paper{8}
\jmonth{Month}
\jname{\textit{IEEE Annals of the History of Computing}}
\jtitle{\textit{IEEE Annals of the History of Computing}}
\pubyear{2025}

\newtheorem{theorem}{Theorem}
\newtheorem{lemma}{Lemma}


\setcounter{secnumdepth}{0}

\begin{document}

\sptitle{ARTICLE}

\title{The First Fifteen Years of Computing in Aotearoa New Zealand}

\author{Brian E. Carpenter}
\affil{University of Auckland, New Zealand}

\author{Sathiamoorthy Manoharan}
\affil{University of Auckland, New Zealand}

\author{Janet Toland}
\affil{Victoria University of Wellington, New Zealand}


\markboth{ARTICLE}{ARTICLE}

\begin{abstract}This article outlines the first fifteen years of modern computing in New Zealand ... 
\end{abstract}

\maketitle


\chapteri{M}odern computers arrived in Aotearoa New Zealand in 1960, little over a century after the first discernable information technology \cite{pioneers}. New Zealand was then a relatively isolated and small (2.5 million people) economy, so it is feasible to track the dissemination and socio-economic influence of computing for the following period; this study runs from 1960 to about 1975. Two events in the mid-1970s signaled the end of the economic and social era that began after World War II, and act as the end of our study: Britain joined the European Common Market in 1973, with major economic consequences, and New Zealand switched on a national Police Computer in 1976, with significant social impact.

In 1960, New Zealand had a rather centrally managed economy. Indeed, its currency did not float until as late as 1985. The details are complex \cite{Sullivan2013}, but the result was that during the period of our study the Treasury was constantly concerned about the foreign exchange impact of computer imports, and this necessitated a system of import licensing. This was a constant background for trends in computing, especially since the Treasury initially favored computing service bureaus to ensure maximum usage of what they considered a scarce resource. Of course, many companies much preferred to have their own systems.

Several themes are used to organize this article:

\begin{itemize}
\item growth in numbers of computers, vendors and employment
\item key areas  and types of usage, including service bureaus
\item start of computing services and teaching in universities, technical colleges
\item professionalization
\item local contributions to the technology
\item commercial and economic impact
\item social impact
\end{itemize}

\vspace*{-8pt}
\section{GROWTH IN NUMBERS AND EMPLOYMENT}

The following table shows the estimated number of electronic digital computers installed in New Zealand over the years in question.

\begin{center}
\begin{tabular}{ |c|c|c|c| } 
 \hline
 Year & Computers & Source & Notes\\ 
\hline
1960 & 2 & \cite{FirstCinNZ}&\\ 
1965 & 70 & \cite{Beardon} & \cite{HoneHeke} gives 45\\ 
1966 & 81 & \cite{Beardon}&\\
1968 & 120 & \cite{Beardon} & \cite{Yearbook75} gives 87\\
1969 & 140 & \cite{Beardon}&\\
1971 & 180 & \cite{Beardon}&\\
1972 & 200 & \cite{Yearbook75}&\\
1974 & 280 & \cite{Beardon}&\\
1976 & 400 & \cite{Beardon}&\\
 \hline
\end{tabular}
\end{center}

Although there are some discrepancies in the available data, the trend is one of rapid growth, from one computer for every 1.2 million people to one for every 8000. Similarly, the number of programmers in public services rose from a handful (maybe 4) in 1960 to 115 (84 men and
31 women) by 1974 \cite{Yearbook75}. Overall, 4000 data processing staff were then employed at  220 sites -- 39\% in Wellington, 31\% in Auckland, 9\% elsewhere in the North Island, and 21\% in the South Island \cite{Beardon}.

Even if the public services had 27\% of female staff, it is not clear that the same applied elsewhere. For example, a staff photograph in 1966 shows 14 men and only three women (Figure~\ref{Staff1966}). Whenever staff appear in computer room photographs, the women are almost invariably seated at a keyboard and the men are standing in discussion.

There is little to indicate significant Māori involvement in computing during our period of study.
Robyn Kāmira \cite{Kamira} shows that such involvement would only start in the 1980s, when a first handful
of Māori information technology specialists graduated.

\begin{figure}
\centerline{\includegraphics[width=18.5pc]{FletcherStaff1966.jpg}}
\caption{\label{Staff1966}Fletcher Computing Bureau staff, 1966. Courtesy Fletcher Archives, NZ.}\vspace*{-5pt}
\end{figure}

New Zealand had a long history of preferring to import British products of all kinds, but this did not long survive in the computer market. IBM was of course a major player; not only did they snag the Treasury as their first customer for an IBM 650 in 1960 \cite{FirstCinNZ}, but they also delivered a heavily discounted IBM 1620 to the University of Canterbury in 1962, the first machine in academia \cite{Dale-Canty2}. This was in line with IBM's world-wide practice of educational discounts, a powerful marketing tool that exposed many students to the IBM brand. As time went on, IBM sold numerous machines - more IBM 1620s, at least one IBM 1401, several IBM 1130s, until the game-changing IBM 360 series came along in 1964.

The main British competitor in 1960 was ICT, but a potential customer would certainly have been confused by the fragmented state of the British computer industry at that time \cite{Hendry-1989}.
Nevertheless, ICT sold one ICT 1201, at least six ICT 1301s, and numerous machines from the ICT 1900 series.
It was not until the various British companies were unified as ICL in 1968 that things settled down, but by then IBM had the lion's share of the New Zealand market and ICL was too far behind. In 1974, the new ICL 2900 series had to compete with the well established and upwards compatible IBM 360 series.

Other companies that were active in the early days were Burroughs (with its E2000, not a general-purpose computer, but specialized for accounting), and NCR (Series 500 , also specialized for ledgers).

TBD? Give a table summarising what we've collected in a spreadsheet. But at best it will be a sample of the market.

Throughout the period under study, most computers remained large and comparatively heavy, even as they moved from ``first generation'' (vacuum tubes) through ``second generation'' (discrete transistors) and ``third generation'' (integrated circuits) to the ``fourth generation'' (very large scale integration). As the technology got smaller, mainframe computers became more powerful rather than more compact; space and air conditioning requirements did not substantially change. For example, the third computer installed in 1969 by Motor Specialties in Auckland, an ICL 1902A, needed to be lifted to the top floor by crane, like its predecessors, an ICT 1201 and an ICT 1301 (Figure~\ref{MS1902A}).

At the same time, the growth in installations led to a minor building boom of dedicated computer centers, which tended to be stark concrete buildings with few windows (Figure~\ref{TAB-ChCh}).

\begin{figure}
\centerline{\includegraphics[width=18.5pc]{MotorSpec1902A-1969.jpg}}
\caption{\label{MS1902A}ICL 1902A Delivery in 1969. Courtesy Fletcher Archives, NZ.}\vspace*{-5pt}
\end{figure}

\begin{figure}
\centerline{\includegraphics[width=18.5pc]{TAB-Chch-1970.jpg}}
\caption{\label{TAB-ChCh}The TAB Computer Centre, Christchurch, 1970. Courtesy Fletcher Archives, NZ.}\vspace*{-5pt}
\end{figure}
TBD: summary and link to next section
\vspace*{-8pt}
\section{HOW COMPUTERS WERE USED}

In 1973, the Department of Statistics provided the following summary \cite{Yearbook73}:

\begin{quotation}
\noindent{The demand for computers has come from Government departments, local authorities, universities, primary producer boards, private firms in industries such as printing, forestry, insurance, oil, food processing, electrical equipment manufacturing, building and construction, clothing, engineering, airways, banking, retailing, motor assembly, paint manufacturing, and stock and station agents.}
\end{quotation}

This was quite a change from 1960, when the demand was for handling the payrolls for teachers and public servants. The expansion in fields of use came very quickly. By the end of 1966 \cite{Treasury-1974} [TBD: Janet check the Archives citation], Government usage included payroll, statistics, science, NZ Railways, Inland Revenue, Post Office, Social Security, Reserve Bank, Bank of New Zealand, New Zealand Broadcasting Corporation, Apple \& Pear Board, and the Dairy Board. In a surviving list of import license applications between August 1965 and May 1966 \cite{Treasury-1974}, there are three applications from local government, three from universities, but 30 from business and primary and secondary industries. The New Zealand economy took to computers rather quickly, held back mainly by import license restrictions.


TBD: types of usage, including service bureaus

TBD: short case studies (EDS, Databank...)

\vspace*{-8pt}
\section{ACADEMIA AND PROFESSIONALIZATION}

TBD: Start of computing services and teaching in universities, technical colleges (and maybe schools?)

TBD: Start of Computer Science as a discipline

TBD: Origin and growth of NZCS

\vspace*{-8pt}
\section{LOCAL CONTRIBUTIONS}

 TBD: Local contributions to the technology 

\vspace*{-8pt}
\section{ECONOMIC IMPACT}

TBD: in the context of the NZ economy decoupling from the UK

\vspace*{-8pt}
\section{SOCIAL IMPACT}

TBD: in the context of increasing privacy concerns and the Wanganui computer

\vspace*{-8pt}
\section{CONCLUSION}

TBD

\vspace*{-8pt}
\section{ACKNOWLEDGMENTS}

TBD

%% style ieeetr doesn't display all details or alphabetise
%% style IEEEtran doesn't alphabetise
%% style plain doesn't meet IEEE requirements
\bibliographystyle{IEEEtranS}
\bibliography{growth}

\begin{IEEEbiography}{Brian E. Carpenter} is an Honorary Professor in the School of Computer Science, University of Auckland, New Zealand. He is interested in Internet protocol design and in computing history. He received a Ph.D. from the University of Manchester, U.K. and is a past Chair of the Internet Engineering Task Force. Contact: brian@cs.auckland.ac.nz\vspace*{8pt}
\end{IEEEbiography}

\begin{IEEEbiography}{Sathiamoorthy Manoharan} is a Senior Lecturer in the School of Computer Science, University of Auckland, New Zealand.
He is a senior member of IEEE. He is interested in computer systems, particularly with respect to performance and security. Contact: mano.manoharan@auckland.ac.nz.\vspace*{8pt}
\end{IEEEbiography}

\begin{IEEEbiography}{Janet Toland} is currently an Associate Professor in the
School of Information Management, Te Herenga Waka, Victoria
University of Wellington, Wellington, New Zealand. Her
research work is in the history of information systems. She
has a particular interest in social and ethical issues. She is
currently the Historian for the Association of Information
Systems. Contact her at Janet.Toland@vuw.ac.nz.\vspace*{8pt}
\end{IEEEbiography}



\end{document}

